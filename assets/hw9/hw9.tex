\documentclass[11pt]{article}
\usepackage{amsmath,amsfonts,amssymb}
\usepackage{hyperref}
\hypersetup{
	colorlinks=true,
	linkcolor=blue,
	urlcolor=cyan
	}
\usepackage[margin=1in]{geometry}
\usepackage{graphicx}

\setlength{\parindent}{0pc}
\setlength{\parskip}{10pt}

\title{STAT157 HW 9}
\date{March 15, 2023}

\begin{document}

\maketitle

\hfill \textbf{Due Tuesday, March 21 at 11:59pm}




\section*{Deliberate Practice: Information Hygiene}

\emph{Expected completion time: 80 minutes}

Read a new article, blog post, or other kind of media that you would normally read and analyze the ``lifetime" of the information presented in that media, with the following components:
\begin{enumerate}
	\item Creation: How was the information in the piece created? Some example questions to ask yourself: If there was an experiment or survey, what was the methodology? If there was an interview, how were the interviewees selected and why might they have agreed to do an interview? 
	\item Transfer: How did this information find its way to me? Some example questions to ask yourself: Does my newsfeed filter out certain articles? Did a friend share this article, and why? Did the editor of the piece try to make the headline more attention-worthy, or otherwise affect whether the piece would make it onto my reading list?
	\item Digestion: How did I handle this information, after reading it? Some example questions to ask yourself: Did I accept the information at face value? Did I dismiss it immediately? Did I want to look up more evidence, either supporting the information or not? How would I expect someone I disagree with to react to this information?
	\item Writeup: In your writeup, include the title of the piece you read and your analysis of the above three components. Also include your overall takeaways from this exercise. Your writeup should be between 400 and 800 words in total.
\end{enumerate}

On Gradescope, please also submit the time it took to complete this exercise.


\section*{Predictions}

\emph{Expected completion time: 90 minutes}

Register the following predictions. You can submit them by going to \href{https://forms.gle/SWyxCaJK3ZxNcmgUA}{this URL} and following the form's instructions. For these predictions, (and all predictions about the future throughout this class), we encourage you to use external sources -- by googling things, reading news articles, talking to friends who follow politics or music stats, etc.

\begin{enumerate}
    \item What will the San Joaquin 5-station \href{https://cdec.water.ca.gov/reportapp/javareports?name=PLOT_FSI.pdf}{precipitation index} be at the end of March?
    \begin{itemize}
		\item The question will resolve based on the sum of the ''Observed(Inches)" column of  \href{https://cdec.water.ca.gov/reportapp/javareports?name=TAB_FSI.pdf}{this table} on March 31 11:59 PM PST. Please submit your answers in inches.
	\end{itemize}
	\item How many novel FDA drug approvals will there have been in 2023 on April 15? 
	\begin{itemize}
		\item The question will resolve as the number of rows on \href{https://www.fda.gov/drugs/new-drugs-fda-cders-new-molecular-entities-and-new-therapeutic-biological-products/novel-drug-approvals-2023}{this table} on April 15.
	\end{itemize}
	\item Will the US lift the \href{https://www.cdc.gov/coronavirus/2019-ncov/travelers/proof-of-vaccination.html}{COVID-19 vaccine mandate} for foreign visitors by EOD April 21?
    \begin{itemize}
		\item The question will resolve positively if, by EOD April 21, the CDC announces on its official website that non-citizens and non-immigrants from any country will no longer need to be fully vaccinated against COVID-19 to enter the US.
	\end{itemize}
\end{enumerate}

For each question, submit an inclusive 80\% confidence interval or your probabilities, as well as an explanation of your reasoning (1-2 paragraphs). \textbf{Please include a copy of your google form responses with your Gradescope submission.} On Gradescope, please also submit the time it took to complete this exercise.

\section*{Extra Credit}

You can earn extra credit for doing the following task:
\begin{enumerate}
\item (2 points). Propose a prediction question for a future homework. Your question should have a fully specified resolution criteria, 
      such that it would be clear to the course staff how to resolve it.
\end{enumerate}

You will get extra credit if we either use your question on a future homework, 
or like it enough to add it to our question bank for future classes.


\end{document}
