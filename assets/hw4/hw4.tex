\documentclass[11pt]{article}
\usepackage{amsmath,amsfonts,amssymb}
\usepackage{hyperref}
\hypersetup{
	colorlinks=true,
	linkcolor=blue,
	urlcolor=cyan
	}
\usepackage[margin=1in]{geometry}
\usepackage{graphicx}

\setlength{\parindent}{0pc}
\setlength{\parskip}{10pt}

\title{STAT157 HW 4}
\date{Feb 7, 2023}

\begin{document}

\maketitle

\hfill \textbf{Due Tuesday, February 14 at 11:59pm}

\section*{Deliberate Practice}

\subsection*{Part a: Forecasting Videos}

\emph{Expected completion time: 40 minutes}

Watch the seven short video clips \href{http://www.stat157.com/assets/initial_clips.zip}{here}. For each of them, spend a couple minutes brainstorming as many different 
possible continuations of the video as you can think of that are qualitatively distinct. After watching each video (and before watching the next one), look at the 
full version of the corresponding video \href{http://www.stat157.com/assets/full_videos.zip}{here}. Was the actual outcome covered by your list?

The idea is to practice the MECE Principle and exhaust the space of possible outcomes. You will probably find that on some of the initial videos, the actual outcome wasn't on your list. 
Your goal is to have the actual outcome be consistently on your list by the end.

As a benchmark, when the instructors did this, they ended up writing around 7 possibilities on average per video (but it's fine if you end up writing more or fewer).

% as much as possible; ideally the outcomes you list cover at least $80$-$90\%$ of the total outcome probability mass.

\subsection*{Part b: Generating Considerations}

\emph{Expected completion time: 40 minutes}

Now we'll apply a similar brainstorming skill to predictions. 
For each of the following three predictions from your homeworks, brainstorm 5 considerations for each that could have substantially affected the final outcome of the forecast.

For example, recall the question ``How many people will be in the Zoom lecture at 11:15am on Wednesday, January 25th?'' In this case, a consideration that affected the outcome is that ``some students deliberately tried to decrease the number of people in the zoom room at that time by staying outside of the room, to have a more accurate prediction than other students''. Or recall the question ``Will Joe Biden resign during his first term?''. A potentially important consideration that could substantially affect the outcome is if Biden resigns due to poor health. Try to generate 5 considerations like these for each of the following:

\begin{enumerate}
	\item[1.] (Homework 1, Q3) How many dollars will the movie \emph{Avatar: The Way of Water} gross in the US on the weekend of Jan 27-29?

	\item[2.] (Homework 3, Q1) Will \href{https://en.wikipedia.org/wiki/You_(TV_series)}{\emph{You}} be among the Top 5 Netflix Shows for the week of February 6 - February 12, according to the \href{https://top10.netflix.com/tv}{Top Ten Netflix TV Shows} ranking (English category)? 

	\item[3.] (Homework 4, Q1) How many NYT headlines/ledes in the \href{https://www.nytimes.com/section/technology}{technology section} dated between Feb 15 and Feb 21 (included) will include the terms "chatbot" or "ChatGPT"?

\end{enumerate}

Afterwards, put a ``Y'' next to each consideration that seems plausibly important to you (greater than 2\% chance of occurring and affecting the answer), 
and ``N'' next to each consideration that does not seem plausibly important.

On Gradescope, please submit the time it took to complete this exercise.

\section*{Lab}

\emph{Expected completion time: 120 minutes}

\href{http://datahub.berkeley.edu/hub/user-redirect/git-pull?repo=https%3A%2F%2Fgithub.com%2Fjs-d%2Fforecasting-class-sp23&urlpath=tree%2Fforecasting-class-sp23%2Fhw%2Fhw4%2Fhw4lab.ipynb&branch=main}{Link to Jupyter notebook.}

Please follow the instructions in the notebook to print out your code and answers and submit to Gradescope. You may use languages other than Python, although we will generally be providing starter code in Python.

On Gradescope, please also submit the time it took to complete this exercise.

\section*{Predictions}

\emph{Expected completion time: 120 minutes}

Register the following predictions. You can submit them by going to \url{https://docs.google.com/forms/d/19mT6vqimX5N-cT7LhbuEAhlQRm3Ma8Rd5O-yKIW2d4c/edit} and following the form's instructions. For these predictions, (and all predictions about the future throughout this class), we encourage you to use external sources -- by googling things, reading news articles, talking to friends who follow politics or music stats, etc.

\begin{enumerate}
	\item[0.] Pick the same website, application, or software you chose last week, and predict how much time you will spend on it between 12:00am Tuesday February 15th and 11:59pm Sunday February 20th., as measured by your time-tracking app.
	\item[1.] How many NYT headlines/ledes in the \href{https://www.nytimes.com/section/technology}{technology section} dated between Feb 15 and Feb 21 (included) will include the terms "chatbot" or "ChatGPT"?
	\item[2.] Of the current forecasters in the top 20 on the leaderboard today, how many will have one of the 20 best average scores on prediction question 1 from this homework?
	\item[3.] Will the \href{https://www.speedrun.com/mc?h=Any_Glitchless-Random_Seed-1.16&x=mkeyl926-r8rg67rn.21d4zvp1-wl33kewl.4qye4731}{speedrunning record} for Minecraft (Java Edition) be beaten by April 20th?
\end{enumerate}

For each question, submit a mean and inclusive 80\% confidence interval (or probabilities for question 3), as well as an explanation of your reasoning (1-2 paragraphs). When detailing your reasoning, please explain \textbf{how you handled your uncertainty}. For example, for the confidence interval questions, how did you choose your upper and lower bound, and why did this seems like the right width for an 80\% interval?

\textbf{Please include a copy of your google form responses with your Gradescope submission.} On Gradescope, please also submit the time it took to complete this exercise.


\end{document}
