\documentclass[11pt]{article}
\usepackage{amsmath,amsfonts,amssymb}
\usepackage{hyperref}
\hypersetup{
	colorlinks=true,
	linkcolor=blue,
	urlcolor=cyan
	}
\usepackage[margin=1in]{geometry}
\setlength{\parindent}{0pc}
\setlength{\parskip}{10pt}

\title{STAT157 HW 2}
\date{Jan 24, 2022}

\begin{document}

\maketitle

\hfill \textbf{Due Monday, January 31 at 11:59pm}

\section*{Deliberate Practice: Calibration}

\emph{Expected completion time: 30 minutes}

Go to \url{https://www.openphilanthropy.org/calibration} and complete 30 questions under the Confidence Intervals (level 60\%) category.

At the end, go to the ``Results'' tab and take screenshots showing that 
you completed at least 30 questions and your calibration 
performance (the chart that appears below the results).

\section*{Deliberate Practice: Estimation}

\emph{Expected completion time: 70 minutes}

For each of the following questions, estimate the answer without looking things up, then look up the answer and calculate the relative error of your estimate\footnote{Note that some of these questions still leave some ambiguity depending on specific definitions, and different Google search results might give slightly different answers! For these quantities, the different results should be pretty similar, so use the first reasonable result you find.}. 


\begin{enumerate}
	\item How many people are enrolled full-time (either 4-year or 2-year) in colleges in the U.S. now? 
	\item How many cattle are in the world? 
	\item What's the length of the SF Bay Bridge, in meters? 
	\item How heavy, in tons, is the Titanic? 
	\item How many TV viewers were there for the 2021 Oscars, according to Nielsen Data? 
	\item How many daily active users watch Twitch? 
	\item How many statistics degrees (Bachelors plus Masters plus PhD) were given in 2020? 
	\item What's the GDP of Mexico 2021? 
	\item How many musicians are employed in the U.S. 
	\item What's the length of the lowest hole, in meters, humans have dug? 
\end{enumerate}

Also, devise two estimation questions of your own, for quantities you are interested in. Again, estimate the answer without looking things up, then look up the answer.

On Gradescope, for each of the \textbf{12 questions (10 provided plus 2 created by you)}, submit your estimate, an explanation of your estimate, and your relative error.


\section*{Lab}

\emph{Expected completion time: 60 minutes}

\href{https://datahub.berkeley.edu/hub/user-redirect/git-pull?repo=https%3A%2F%2Fgithub.com%2Fjs-d%2Fstat-157-260-website&urlpath=tree%2Fstat-157-260-website%2Fhw%2Fhw2%2Fhw2_lab.ipynb&branch=main}{Link to Jupyter notebook.} 

Please follow the instructions in the notebook to print out your code and answers and submit to Gradescope. You may use languages other than Python, although we will generally be providing starter code in Python.



\section*{Predictions}

\emph{Expected completion time: 35 minutes}

Register the following predictions. You can submit them by going to \url{https://forms.gle/o89xw3s2iEr252KK6} and following the form's instructions.

\begin{enumerate}
	\item[0.] Pick the same website, application, or software you chose last week, and predict how much time you will spend on it between 12:00am Tuesday February 1st and 11:59pm Sunday February 6th., as measured by your time-tracking app.
	
	\item[1.] How many percents of the vote will Figueres get in the first round of the Costa Rican presidential \href{https://en.wikipedia.org/wiki/2022_Costa_Rican_general_election}{election} on Feb 6?
	
	\item[2.] Will Adele's \textit{Easy on me} top the \href{https://www.billboard.com/charts/hot-100/}{Billboard Hot 100} on the week of Feb 5?
 
	\item[3.] Will Russia invade Ukraine between Feb 1 and Feb 7?
	
	
\end{enumerate}

For each question, submit a mean and inclusive 80\% confidence interval (or a probability for questions 2 and 3), as well as an explanation of your reasoning (~1-2 paragraphs).


\end{document}