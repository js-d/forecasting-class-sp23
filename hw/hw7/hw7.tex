\documentclass[11pt]{article}
\usepackage{amsmath,amsfonts,amssymb}
\usepackage{hyperref}
\hypersetup{
	colorlinks=true,
	linkcolor=blue,
	urlcolor=cyan
	}
\usepackage[margin=1in]{geometry}
\usepackage{graphicx}

\setlength{\parindent}{0pc}
\setlength{\parskip}{10pt}

\title{STAT157 HW 7}
\date{March 1, 2022}

\begin{document}

\maketitle

\hfill \textbf{Due Monday, March 7 at 11:59pm}

\section*{Deliberate Practice: Prioritizing Information}

\emph{Expected completion time: -- minutes}

In this deliberate practice, produce a forecast for the following question:
\begin{quote}
	What will the price of crude oil be on March 8, 11:59 pm PT? (Specifically WTI Crude Futures)
\end{quote}
using techniques from lecture, by following these steps:

\begin{enumerate}
	\item Spend \textbf{10 minutes} brainstorming key considerations that would affect your forecast. 
	\item Spend \textbf{15 minutes} assigning ratings to each of the considerations for importance, uncertainty, and how quickly you can resolve the uncertainty (as in lecture 11).
	\item Using those ratings, rank the considerations in order of priority to reduce uncertainty. Spend \textbf{30 minutes} using Google/other resources to reduce uncertainty on those top ranked considerations. 
	\item Spend \textbf{10 minutes} to re-evaluate the uncertainty on your considerations.
	\item Spend \textbf{20 minutes} writing up this exercise. Please include:
	\begin{itemize}
		\item Your considerations, ratings, which considerations you chose to research, and how much uncertainty you reduced using external sources.
		\item Your final point estimate (mean estimate) for the forecast.
		\item Reflections on the exercise -- were some considerations harder than expected to research? If you were to do this again, would you have chosen different considerations to research with your 30 minutes?
	\end{itemize}
 Submit this writeup to Gradescope.
\end{enumerate}  


On Gradescope, please also submit the time it took to complete this exercise.

\section*{Deliberate Practice: Structural vs. Numerical Uncertainty}


\emph{Expected completion time: -- minutes}

Using the forecast you created for the above question, assess the uncertainty of your point estimate with the following steps:
\begin{enumerate}
	\item Assess structural uncertainty:
	\begin{itemize}
		\item Brainstorm 2 considerations relevant for structural uncertainty, i.e. considerations that could cause your previous estimate to be totally off (see Lecture 12 for examples).
		\item For each of these considerations, quantify how much they would change your estimate, and quantify the probability that these considerations turn out to be true and relevant.
		\item Based on this, create an 80\% confidence interval around your original point estimate.
	\end{itemize}
	
	\item Assess numerical uncertainty:
	\begin{itemize}
		\item Choose 2 of the considerations you generated above in Deliberate Practice: Prioritizing Information, and assess the sensitivity of your forecast to uncertainty in these considerations.
		\item Based on this, create an 80\% confidence interval around your original point estimate.   
	\end{itemize}
	
	\item Combine the two sources of uncertainty into a final 80\% confidence interval for the forecast question.
	\item Write up your considerations, quantifications, and reflections on the final confidence intervals you produced.
\end{enumerate}


On Gradescope, please also submit the time it took to complete this exercise.

\section*{Predictions}

\emph{Expected completion time: 120 minutes}

Register the following predictions. You can submit them by going to TODO:URL and following the form's instructions. For these predictions, (and all predictions about the future throughout this class), we encourage you to use external sources -- by googling things, reading news articles, talking to friends who follow politics or music stats, etc.

\begin{enumerate}
	\item Will the \href{https://www.elle.com/horoscopes/daily/a107/aquarius-daily-horoscope/}{daily horoscope} on Elle.com on Tuesday March 8 contain more than 10 copies of the word ``you''?
	\item What will be the \href{https://www.rottentomatoes.com/m/the_batman}{Rotten Tomatoes} audience score for \emph{The Batman} as measured on April 15?
	\item When will hyperpop duo \href{https://en.wikipedia.org/wiki/100_Gecs}{100 gecs} release their new album, 10000 gecs?
	\begin{itemize}
		\item This question resolves based on whether the album will be available in full on Spotify.
		\item If the album is not available by April 20, then this question will resolve as April 20.
	\end{itemize}
	\item What will be the price of peanut butter at Andronico's on April 2?
	\begin{itemize}
		\item To resolve this question, I will go to Andronico's (the store on 1550 Shattuck Ave) on April 2 and ask for the price of peanut butter.
		\item TODO: clarify whether Andronico's has its own homemade peanut butter, in which case I'll ask for the price of 1lb of that, otherwise I'll specify a specific brand + volume.
	\end{itemize}
	\item Who will win the \href{https://bnpparibasopen.com/players/current/?assoc=wta&type=singles}{Indian Wells} Women's Singles tournament? Give a probability that it's Ashleigh Barty, Barbora Krejcikova, Aryna Sabalenka (the current WTA top 3), Paula Badosa (last year's winner), or someone else.
\end{enumerate}

For each question, submit a mean and inclusive 80\% confidence interval or your probabilities, as well as an explanation of your reasoning (1-2 paragraphs). \textbf{Please include a copy of your google form responses with your Gradescope submission.} On Gradescope, please also submit the time it took to complete this exercise.

\section*{Extra Credit}

You can earn extra credit for doing one or both of the following tasks:
\begin{enumerate}
\item (2 points). Propose a prediction question for a future homework. Your question should have a fully specified resolution criteria, 
      such that it would be clear to the course staff how to resolve it.
\item (2 points). Write down 5 ``calibration-app'' style questions (and answers), for a domain that is not trivia.
\end{enumerate}

For the first part, you will get extra credit if we either use your question on a future homework, 
or like it enough to add it to our question bank for future classes.

For the second part, you will get extra credit if we think they are at least as interesting as the 
questions currently in the calibration app, and well-written enough to be used as a calibration question. 
(We may submit them to the app creator and suggest they include them!)

\end{document}
